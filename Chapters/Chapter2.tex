\chapter{Part Two}
During the first year of my Ph.D. study, I have taken Disciplinary, Multi-disciplinary and Specialized research management courses. This courses/activities helped me to satisfy my credit requirements and gain knowledge in my field.

In addition to the courses, I have attended multiple seminar/ talks in Istituto Italiano di Tecnologia (IIT), a more detailed summer about the speakers and the title of the talk are listed in Table \ref{table:Seminar}.

\section{Disciplinary and Multi-disciplinary Courses/Activities}

\textbf{1. PhD School at the University of Verona:}\\
Title: "Advanced Topics in Deep Learning"\\
Date:  May 27 - 31, 2019.\\
Teachers: Dr. Timothy M. Hospedales, Dr. Henry Gouk  and Dr. Francesco Setti.\\
Number of hours: 21.5 hours.\\ 
Credits: 5.4 CFUs.

\textbf{2. Courses at the Istituto Italiano di Tecnologia (IIT):}\\
Title: "Introduction to Physical human-robot interaction." \\
Date: January 14 - 18, 2019.\\ 
Teacher: Dr. Jacopo Zenzeri, PhD. \\
Number of hours: 12.\\
Credits: 4 CFUs. 

\textbf{3. Courses at the University of Genova:}\\
Title: "Interaction in Virtual and Augmented Reality".\\
Date: September 3 - 6, 2019. \\
Teacher: Prof. Manuela Chessa. \\
Number of hours: 12 hours. \\
Credits: 4 CFUs. 

\textbf{4. Courses from Coursera Google Brain:}\\
Title: "Introduction to TensorFlow for Artificial Intelligence, Machine Learning, and Deep Learning". \\ 
Teacher: Laurence Moroney, AI Advocate.\\
Number of hours: 20 hours.\\
Credits: 5 CFUs.

\textbf{5. Courses from Coursera Google Brain:}\\
Title: "Convolutional Neural Networks in TensorFlow". \\ 
Teacher: Laurence Moroney, AI Advocate.\\
Number of hours: 20 hours.\\
Credits: 5 CFUs. \\

The list of seminars and talks I attended are listed in the Table \ref{table:Seminar} With an average hour of one hour and 30 minutes each. I attended a total of 10 seminars with a total hours of 15 with 3.75 CFUs.
\begin{table}[ht]
\resizebox{\textwidth}{!}{ \begin{tabular}{p{2.5cm} p{14cm}}
     \hline
        \textbf{ Date} & \textbf{Title and Speaker}  \\
        \hline
        \begin{center}\
            January 24, 2019 
        \end{center} & 
            "Control schemes for safe physical Human-Robot Interaction" by Prof. Alessandro De Luca – Dipartimento di Ingegneria Informatica Automatica e Gestionale (DIAG), Sapienza, Università di Roma \\
       \rowcolor{lightgray}\begin{center}
           July 11, 2019
       \end{center} &  "Complexity trade-offs for robot motion and manipulation skills" by Prof. Ville Kyrki, Associate Professor, Department of Electrical Engineering and Automation, Aalto University, Aalto, Finland \\
       
       \begin{center}
           July 16, 2019
       \end{center}& "Insight into scientific publishing: career options offered in the industry and how scientific journals operate and evolve." by Paul-André Genest, Ph.D., Senior Editor, Research Publishing, North America - Wiley, 111 River Street , Hoboken NJ 07030 \\
       
       \rowcolor{lightgray} \begin{center}
           February 7, 2019
       \end{center}& “Screw Theory and Its Application in Robotics” by Marco Carricato, Professore Associato - Dipartimento Ingegneria Industriale – Università di Bologna \\
        
        \begin{center}
            September 6, 2018
        \end{center} & "A Special Road of Bringing Consumer Robots to the World - UBTECH Robotics Corp" Jianxin Pang, VP of UBTECH Robotics Corp., Executive Director of UBTECH Research, Shenzhen, Guangdong, China\\
        
       \rowcolor{lightgray} \begin{center}
           June 19, 2019
       \end{center} &“Humanoid Robot Technology in HUBO Laboratory” Prof. Jun Ho Oh: “Humanoid Robot Technology”\\
        \begin{center}
            March 29, 2019
        \end{center}& "Deep Learning in Robotics" by Norman Di Palo - Researcher, Rome, Italy \\
        
        \rowcolor{lightgray} \begin{center}
            November 29 2018
        \end{center} & "Building Small Robots" by Justin Yim, PhD student, Electrical Engineering and Computer Sciences Department, University of California, Berkeley, USA \\
        \begin{center}
            September 10, 2019 
        \end{center} & "Metal-Insulator-Metal Plasmonic Colour Filters for Multispectral Imaging Applications" by Nadia Pinton, Department of Engineering Science, University of Oxford, and School of Engineering Science, University of Glasgow\\
        
        \rowcolor{lightgray} \begin{center}
              September 10, 2019
        \end{center} & "Interactive (machine) learning: a key component of the HMI of the future" by Claudio Castellini, German Aerospace Center, Wessling, Germany\\
 \end{tabular}}
 \caption{List of seminar and Talks I attended}
 \label{table:Seminar}
\end{table}

\section{Courses related to research management}

\textbf{1. Course by Stanford University from Coursera: }\\
Title: "Writing in the Sciences"\\
Teachers: Dr. Kristin Sainani \\
Date: March 5 - April 5, 2019 \\
Number of hours: 32 hours.\\ 
Credits: 8 CFUs.

\textbf{2. Course by University of London from Coursera:}\\
Title: "Understanding Research Methods!"\\
Date:  May 27 - 31, 2019.\\
Teachers:  Dr J. Simon Rofe and  Dr Yenn Lee \\
Number of hours: 15 hours.\\ 
Credits: 3.75 CFUs.